%%%%%%%%%%%%%%%%%%%%%%%%%%%%%%%%%%%%%%%%%%%%%%%%%%%%%%%%%%%%%%%%%%%%%%%%%%%%%%%%
%2345678901234567890123456789012345678901234567890123456789012345678901234567890
%        1         2         3         4         5         6         7         8
% THESIS INTRODUCTION

\chapter*{Introduction}
\label{chap:introduction}
\ifpdf
    \graphicspath{{Introduction/Figures/PNG/}{Introduction/Figures/PDF/}{Introduction/Figures/}}
\else
    \graphicspath{{Introduction/Figures/EPS/}{Introduction/Figures/}}
\fi

% quote

%\setlength{\epigraphwidth}{.35\textwidth}
%\epigraph{Research is formalized curiosity.}{ Zora Neale Hurston, 1942}

% examples of sections

\section{Motivations}
\label{motivations}
In recent years, Internet Of Things technologies (IoT) have grown in
importance and are expected to have an exponential increase in their
use in the coming years. For this reason, there is a need for
communication protocols with two key features, which are:
- Long range communications
- Low power consumption.
To cover these needs LPWANs (Low Power Wide Area Networks) were
created and offer big advantages when compared to high bitrate and
shorter range technologies such as Wi-Fi or Bluetooth, and cellular ones
like GSM or 4G that have a much higher power consumption.\\
The study of these technologies on the enviroment of tightly packed 
cities is highly important in order to advance to achive smart cities.
  
\section{Context of the Study}
\label{context}
The study is performed on the city of Genova, thightly packed with 
buildings, for the most part made with concrete and brick. Other 
cities with different architecture could perform different as the 
distribution of the buildings is important in the reach of the 
signal. One example of this is the problems with GPS localization 
when near skyscrapers.

\section{Tools used in the thesis}
\label{objectives}
For the storage and analysis of these messages The Things Network is
being used, which is a free to use LoRa network that offers not only
gateways but also data storage and various functionalities for a
particular network, being able to make it private or public.

\section{Overview of the Thesis}
\label{overview}
In this thesis we are going to take a look at LoRaWAN (Long Range
Wide Area Network), a communication protocol with a low bit rate and
power consumption that can reach distances of up to several kilometres,
we will study the characteristics of the communication and recreate
various real life scenarios to study different characteristics such as
packet loss, power transmitted and the best configuration for the end-
devices. With these measurements we can obtain valuable information
about how to optimize the network for its different use cases, as there
are big differences depending on the topology and placements of the
nodes and gateways.