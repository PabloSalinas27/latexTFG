%%%%%%%%%%%%%%%%%%%%%%%%%%%%%%%%%%%%%%%%%%%%%%%%%%%%%%%%%%%%%%%%%%%%%%%%%%%%%%%%
%2345678901234567890123456789012345678901234567890123456789012345678901234567890
%        1         2         3         4         5         6         7         8
% THESIS CONCLUSIONS
\def\baselinestretch{1}
\chapter{Conclusions and future work}
\label{chap:conclusions}
\ifpdf
    \graphicspath{{Conclusions/Figures/PNG/}{Conclusions/Figures/PDF/}{Conclusions/Figures/}}
\else
    \graphicspath{{Conclusions/Figures/EPS/}{Conclusions/Figures/}}
\fi
\def\baselinestretch{1.0}

% quote
\section{First experiment conclusion}

In the configuration with ADR the ED did not change from SF7. This is problably because 
we were moving and we didn’t get 20 uplinks a low maximun SNR to increase the SF. We din’t 
get either 64 uplinks without a confirmation as our maximun packet loss between two uplinks 
is 35. (All this is expalined in \ref*{chap4:exp1}). In this case the ADR configuration was 
pointless as the end node wasn´t able to change the parameters like the SF or the power
to obtain the optimal data transmission. As we got always an spreading factor of 7, the performance 
during the experiment with this configuration is worse compared with the one we selected with SF10 
and a fixed Tx power of 14dBm as you can see in \ref*{tab:packet_loss_exp1}. The reason why in \ref*{tab:RSSI_SNR_exp1}
the values of SNR and RSSI are lower in the case of ADR is because the other configuration is able of detecting 
signal in worse situationsas we can see in \\
Thanks to the performance of this experiment we were able to extract 
valuable information releted with the signal received depending on the landscape of the city. 
We distinguish two zones, the area of ‘vicolis’ and the port area:
‘Vicoli’ is an italian word used in the city of Genova for narrow streets. The historic center of Genova 
is plenty of them making the data transmission almost impossible unless the ED
is close to the gateway (around 100m but it depends if you have direct vision). Next to this zone is the 
area of ‘porto-antico’ which is a much more open-space area where the comunnications reach much longer distances.
The results and graphics shows that the port area showed in the figure (link) is much more 
suitable for a LoRa comunnication rather than the area of the 'vicolis’.

\section{Second experiment conclusion}




